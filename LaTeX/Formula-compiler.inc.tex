% -*- TeX:UK -*-
\chapter{Formula compiler}
\begin{refsection}

\abstract{Formula compiler allow the user to enter a mathematical formula at run-time, which is evaluated by the program. Thus, programs can fulfill many different functions without the need of recompilation. This compiler can also perform symbolic differentiation. }

As the unit in \parencite{Gie-87b,Eng-89} was originally written for \acs{CPM}, it required some modernisation, but otherwise is tried and tested.

The function \texttt{CompileExpression(Expr, VarTable, ExprPtr)} used here has the following properties:
\begin{itemize}
  \item{Each formula can contain several variables, whose names can be chosen freely as long as they start with a letter and are not identical to the reserved names (functions) of the compiler. }
  \item{The \Name{Boole}an variable \texttt{CalcDecMode} decides whether new variables entered into a formula are added to the \texttt{VarTable} (which is then \texttt{nil} at the beginning), or whether the expression can contain only such variable as are already in the \texttt{VarTable}.}
  \item{The syntax of the expression string is the same as in \texttt{Pascal}, the expression must close with a ``;''. Anything following the ``;'' is ignored. }
  \item{The final compilate is returned in \texttt{ExprPtr}, using reverse polish notation (UPN).}
\end{itemize}
Any such \texttt{ExprPtr} can be evaluated by \texttt{CalcExpression(ExprPtr, VarTbl)}, where \texttt{VarTbl} contains the current values for all variables in the function. The variable \texttt{CalcResult} becomes \texttt{false} upon any errors during compilation or execution.

\texttt{Calc} can be made to handle additional functions:
\begin{enumerate}
  \item{The name of the function needs to be added to \texttt{Calc\_symbols} and \texttt{Calc\_Ids}}
  \item{The actual algoritm is enterd in the \texttt{case}-statement of \texttt{CalcExpression}.}
  \item{The data type \texttt{Calc\_operand} can be changed for example to \texttt{complex}, then all function calls need to be changed to their \texttt{complex} equivalents.}
\end{enumerate}

\section{The source code}

\begin{lstlisting}[caption=Interface]
  unit Calc;

  interface

  uses Crt, MathFunc;

  const
    Calc_IdLen = 10;
    Calc_MaxVar = 10;
    Calc_OpSize = 6;

  type
    ErrString    = string[80];
    Calc_IdStr   = string[Calc_IdLen];
    Calc_String  = string[255];
    Calc_Operand = float;

   Calc_VarType  = record
                     VarId: Calc_IdStr;
                     Value: Calc_Operand;
                   end;

  Calc_VarTab   = ^Calc_VarTable;
  Calc_VarTable = array[0..Calc_MaxVar] of Calc_VarType;
  Calc_Symbols  = (Calc_Err, Calc_EOE, Calc_Const, Calc_Var, Calc_Pi,
                   Calc_E, Calc_lp, Calc_rp, Calc_Neg, CAlc_Add, Calc_Sub,
                   CAlc_Mul, Calc_Dvd, Calc_Div, Calc_Mod, Calc_ggT,
                   Calc_kgV, Calc_Pow, Calc_sqr, Calc_Sqrt, Calc_Exp,
                   Calc_Ln, CAlc_Lg, Calc_Ld, CAlc_Sin, Calc_Cos, Calc_Tan,
                   Calc_Cot, Calc_ArcSin, CAlc_ArcCos, Calc_ArcTan,
                   Calc_ArcCot, Calc_Sinh, Calc_Cosh, Calc_Tanh, CAlc_Coth,
                   Calc_ArcSinh, Calc_ArcCosh, Calc_ArcTanh, Calc_ArcCoth,
                   Calc_Abs, Calc_Deg, CAlc_Rad, Calc_Rez, CAlc_Fak,
                   Calc_Sign, Calc_Int, Calc_End);

  Calc_Prog     = ^Calc_Instruct;

  Calc_Instruct = record
                    NextInst: Calc_Prog;
                    Instruct: Calc_Symbols;
                    case Calc_Symbols of
                      Calc_Var  : (VarIndex: integer);
                      Calc_Const: (Operand: Calc_Operand);
                  end;

  var   CalcDecMod, CalcResult: boolean;


  procedure CompileExpression(Expr: Calc_String; var VarTable: Calc_VarTab;
    var ExprPtr: Calc_Prog);
  { turn arithmetic expressions into UPN }

  function CalcExpression(ExprPtr: Calc_Prog; VarTable: Calc_VarTab): Calc_Operand;
  { calculate the result of an expression }

  function CalcDerivation(pptr: Calc_Prog; VarTab: Calc_VarTab;
            Nach: Calc_IDStr): Calc_Prog;
  { symbolic derivative of expression }

  procedure CalcAOS(pptr: Calc_Prog; VarTable: Calc_VarTab);
  { turn UPN-notation into AOS-formula and display }

  procedure CalcError(ErrNum: integer; Message: ErrString);
  {catch errors }

  procedure KillExpression(var ExprPtr: Calc_Prog);
  { delete Calc_prog if no longer needed }

  function NewVarTab: Calc_VarTab;
  { create new, empty vatiable table}

  procedure KillVarTab(var VarTab: Calc_VarTab);
  { delete variable table if no longer needed }

  function SearchVarTab(VarTab: Calc_VarTab; ID: Calc_String): integer;
  { search a variable in the variable table and return its index }

  function AddToVarTab(VarTab: Calc_VarTab; ID: Calc_String): integer;
  { enter a new variable into the variable table and return its index }

  procedure AssignVarI(VarTab: Calc_VarTab; i: integer; x: Calc_Operand);
  { assign the i-th variable in the table the value x }

  procedure AssignVar(VarTab: Calc_VarTab; ID: Calc_String; x: Calc_Operand);
  { assign the value x to the variable ID in the variable table}

  procedure HelpFormula;
  { produces a short help text }

  implementation

  const
    Calc_Ids: array [Calc_Symbols] of Calc_IdStr =
                 ('ERR', ';', 'CONST', 'VAR', 'PI', 'E', '(', ')', 'NEG',
                  '+', '-', '*', '/', 'DIV', 'MOD', 'GGT', 'KGV', '^',
                  'SQR', 'SQRT', 'EXP', 'LN', 'LG', 'LD', 'SIN', 'COS',
                  'TAN', 'COT', 'ARCSIN', 'ARCCOS', 'ARCTAN', 'ARCCOT',
                  'SINH', 'COSH', 'TANH', 'COTH', 'ARCSINH', 'ARCCOSH',
                  'ARCTANH', 'ARCCOTH', 'ABS', 'DEG', 'RAD', 'REZ', 'FAK',
                  'SGN', 'INT', 'END');
\end{lstlisting}

\subsection{Administrative routines}

This routine catches errors, prints an error message and sets the flag \texttt{CalcResult} to false so that the calling program may handle the situation gracefully:
\begin{lstlisting}
  procedure CalcError(ErrNum: integer; Message: ErrString);

  var
    Meldung: string;
    ch: char;

  begin
    case ErrNum of
      1: Meldung := ' *** Run time error: Floating point overflow' + Message;
      2: Meldung := ' *** Run time error: Division bx zero' + Message;
      3: Meldung := ' *** Run time error: Argument error in ' + Message;
      else
        Meldung := ' *** Run time error: ' + Message;
    end;
    ch := MathFunc.WriteErrorMessage(Meldung);
    CalcResult := False;
  end;
\end{lstlisting}

The following routine removes a Calc-program from the heap, this will be called after failed calls to \texttt{CompileExpression}, and may be called by user programs to avoid heap overflow:
\begin{lstlisting}
  procedure KillExpression(var ExprPtr: Calc_Prog);

  var
    NextPtr: Calc_Prog;

  begin
    while ExprPtr <> nil do
    begin
      NextPtr := ExprPtr^.NextInst;
      Dispose(ExprPtr);
      ExprPtr := NextPtr;
    end;
    ExprPtr := nil;
  end;
\end{lstlisting}

This routine generates a new variable table
\begin{lstlisting}
  function NewVarTab: Calc_VarTab;

  var
    VarTab: Calc_VarTab;

  begin
    Result := nil;
    try
      begin
        New(VarTab);
        VarTab^[0].Value := 0.0;
        Result := VarTab;
      end
    except
      CalcError(0, ' not enough memory');
    end
  end;
\end{lstlisting}

This procedure removes a no longer needed variable table from memory
\begin{lstlisting}
  procedure KillVarTab(var VarTab: Calc_VarTab);

  begin
    if vartab <> nil then
      Dispose(VarTab);
    VarTab := nil;
  end;
\end{lstlisting}

This routine searches for the variable named ID in the variable table. Returns 0 if variable is not found, the index otherwise:
\begin{lstlisting}
  function SearchVarTab(VarTab: Calc_VarTab; Id: Calc_String): integer;

  var
    i: integer;

  begin
    if vartab <> nil
    then
      begin
        for i := 1 to Length(id) do
          id[i] := upcase(id[i]);
        i := Trunc(VarTab^[0].Value);
        VarTab^[0].VarId := Copy(Id, 1, Calc_IdLen);
        while VarTab^[i].VarId <> Id do
          i := Pred(i);
        Result := i;
      end
    else
      Result := 0;
  end;
\end{lstlisting}

This function adds the variable ID to the next free position in the variable table and returns the position if space was available. Otherwise, the return value is \(-1 \).
\begin{lstlisting}[caption=]
  function AddToVarTab(VarTab: Calc_VarTab; Id: Calc_String): integer;

  var
    i: integer;

  begin
    if VarTab <> nil
      then
        begin
          for i := 1 to Length(id) do
            id[i] := upcase(id[i]);
          i := Trunc(VarTab^[0].Value);
          if i < Calc_MaxVar
            then
              begin
                i := Succ(i);
                VarTab^[0].Value := i;
                VarTab^[i].VarId := Id;
                VarTab^[i].Value := 0;
              end
            else
              i := -1;
          Result := i;
        end
      else
        Result := -1;
  end;
\end{lstlisting}

This routine assigns the value x to the variable at position i in the variable table
\begin{lstlisting}[caption=]
  procedure AssignVarI(VarTab: Calc_VarTab; i: integer; x: Calc_Operand);

  begin
    if vartab <> nil
      then
        begin
          if (i > 0) and (i <= Trunc(VarTab^[0].Value))
            then VarTab^[i].Value := x
            else CalcError(0, 'value assigned to unknown variable');
        end
      else
        CalcError(0, 'value assigned to unknown variable');
  end;
\end{lstlisting}

This routine assigns the value x to the variable ID in the variable table
\begin{lstlisting}
  procedure AssignVar(VarTab: Calc_VarTab; Id: Calc_String; x: Calc_Operand);

  begin
    AssignVarI(VarTab, SearchVarTab(VarTab, Id), x);
  end;
\end{lstlisting}

\begin{lstlisting}
  procedure HelpFormula;

  begin
      writeln('The formula is entered in Pascal-syntax and must end with a semicolon.');
      writeln;
      writeln('The compiler ''knows'' the following constants and functions, which ');
      writeln('can not be redefined:');
      writeln('Constants: e, pi                        Basic operators: +, -, *, /, ^');
      writeln('Integer: div, mod, ggt, kgv             Logarithms: ln, lg, ld, exp');
      writeln('sin, cos, tan, cot and the equivalent hyperbolic and arcus functions');
      writeln('Various Functions: abs, deg, rad, fak, sgn');
      writeln;
  end;
\end{lstlisting}



\subsection{Symbolic differentiation}

The formula-string is compiled into a UPN program (\( f(x) = (7 + x)*x^3 \rightarrow @ 7 x + x 3 \) \textasciicircum\ \( \times \), with @ the list anchor). This means, that for symbolic differentiation the routine would find the first operator only after reading over the operands. Therefore, the UPN-program is inverted first. The derivation itself is then performed by the recursive procedure \texttt{derive}.

The result is then simplified by \texttt{CalcSimplify}, which repaces operations with constants by their result, and handles operations with the constants \num{1} and \num{0}. The implementation of the commutative law is still limited. Indeed, when I showed this program to my maths-teacher, he had great fun finding equations whose differential is simple, but result in fairly complex (yet correct!) formulas in \texttt{CalcDerivation}. There still is room for improvement here.

\begin{lstlisting}[caption=]
  procedure invert(pptrstart: calc_prog);

  var
    pptr, pptr1, pptr2: calc_prog;
    max, i: integer;
    dummy: calc_instruct;

  begin
    if pptrstart <> nil
      then
        begin
          pptr := pptrstart^.nextinst;
          max := 0;
          while pptr^.nextinst <> nil do
            begin
              pptr := pptr^.nextinst;
              max := Succ(max);
            end;
          pptr := pptrstart;
          repeat
            pptr := pptr^.nextinst;
            pptr1 := pptr;
            for i := 1 to max do
              pptr1 := pptr1^.nextinst;
            dummy := pptr^;
            pptr^ := pptr1^;
            pptr1^ := dummy;
            pptr2 := pptr^.nextinst;
            pptr^.nextinst := pptr1^.nextinst;
            pptr1^.nextinst := pptr2;
            max := max - 2;
          until max <= 0;
        end; { then }
  end;
\end{lstlisting}

\begin{lstlisting}
  function endof(pptr: calc_prog): calc_prog;

  var
    help: calc_prog;
    op: integer;

  begin
    if pptr <> nil
      then
        begin
          op := 1;
          repeat
            help := pptr;
            if pptr^.instruct in [calc_var, calc_const]
              then op := Pred(op)
              else if not (pptr^.instruct in [calc_neg, calc_sqr..calc_fak])
                 then
                   op := Succ(op);
            pptr := pptr^.nextinst
          until op = 0;
          Result := help;
        end
      else
        Result := nil;
  end;
\end{lstlisting}

\subsection{Simplification of Calc-programs}

\begin{lstlisting}
  procedure CalcSimplify(var pptr: calc_prog);

  var
    pptr1, help1, help2     : calc_prog;
    op                      : integer;
    dummy                   : calc_operand;
    arg1, arg2, SimpleError : boolean;
    helpinstruct            : calc_symbols;


    function Equal(pptr1, pptr2: calc_prog): boolean;

    var
      help1, help2 : calc_prog;
      check        : boolean;

    begin
      help1 := endof(pptr1);
      help2 := endof(pptr2);
      if (pptr1 <> nil) and (pptr2 <> nil) and (help1 <> nil) and (help2 <> nil)
        then
          begin
            check := True;
            repeat
              check := check and (pptr1^.instruct = pptr2^.instruct);
              case pptr1^.instruct of
                calc_const : check := check and (pptr1^.operand = pptr2^.operand);
                calc_var   : check := check and (pptr1^.varindex = pptr2^.varindex)
              end;
              pptr1 := pptr1^.nextinst;
              pptr2 := pptr2^.nextinst
            until not check or (pptr1 = help1^.nextinst) and (pptr2 = help2^.nextinst);
            Result := check;
          end
        else
          Result := False;
    end; // Equal


    function compute(pptr, pptr1, pptr2: calc_prog): calc_operand;

    var
      exptr, a, b, c : calc_prog;
      vardummy       : calc_vartab;

    begin
      try
        begin
          vardummy := newvartab;
          New(a);
          New(b);
          New(c);
          a^ := pptr^;
          b^ := pptr1^;
          if pptr2 <> nil
            then c^ := pptr2^;
          a^.nextinst := nil;
          New(exptr);
          exptr^.nextinst := b;
          if pptr2 <> nil
            then
              begin
                b^.nextinst := c;
                c^.nextinst := a;
              end
            else
              b^.nextinst := a;
          Result := calcexpression(exptr, vardummy);
          SimpleError := SimpleError or not calcresult;
          Dispose(a);
          Dispose(b);
          Dispose(c);
          Dispose(exptr);
          killvartab(vardummy);
        end
      except
          begin
            Result := 0.0;
            SimpleError := True;
          end;
      end;
    end; // Compute


    procedure simple(pptr: calc_prog);

    var
      pptra, pptrb : calc_prog;


      procedure restoreptr;

      begin
        pptra := pptr^.nextinst;
        pptrb := endof(pptra);
        if pptrb <> nil
          then pptrb := pptrb^.nextinst;
      end;


      procedure erase_entry;

      begin
        while help1 <> pptrb do
          begin
            help2 := help1;
            help1 := help1^.nextinst;
            Dispose(help2);
          end;
      end;


      procedure pusha;

      begin
        pptr^ := pptra^;
        help1 := pptr;
        while help1^.nextinst <> pptrb do
          help1 := help1^.nextinst;
        help1^.nextinst := pptrb^.nextinst;
        Dispose(pptra);
        Dispose(pptrb);
        restoreptr;
      end;


      procedure skipa;

      begin
        help1 := pptr^.nextinst;
        erase_entry;
        pptr^ := pptrb^;
        Dispose(pptrb);
        restoreptr;
      end;


      procedure setconst(dummy: calc_operand);

      begin
        pptr^.instruct := calc_const;
        pptr^.operand := dummy;
        pptr^.nextinst := pptrb^.nextinst;
        help1 := pptra;
        erase_entry;
        Dispose(pptrb);
        restoreptr;
      end;

    begin // Simple
      if pptr <> nil
        then
          begin
            restoreptr;
            if pptr^.instruct in [calc_neg, calc_sqr..calc_fak]
              then
                begin
                  simple(pptra);
                  if pptra^.instruct = calc_const // ausrechnen !
                    then
                      begin
                        dummy := compute(pptr, pptra, nil);
                        if dummy = -0.0
                          then dummy := 0.0;
                        pptr^.instruct := calc_const;
                        pptr^.operand := dummy;
                        pptr^.nextinst := pptra^.nextinst;
                        Dispose(pptra);
                        restoreptr;
                      end;
                  if pptr^.instruct = calc_neg
                    then
                      begin
                        if pptra^.instruct = calc_neg
                          then
                            begin
                              pptr^ := pptra^.nextinst^;
                              Dispose(pptra^.nextinst);
                              Dispose(pptra);
                              restoreptr;
                            end;
                      end;
                  if (pptr^.instruct = calc_neg) and (pptra^.instruct in
                    [calc_mul.. calc_div])
                    then
                      begin
                        help1 := endof(pptra^.nextinst);
                        if pptra^.nextinst^.instruct = calc_const
                          then
                            begin
                              pptra^.nextinst^.operand := -pptra^.nextinst^.operand;
                              pptr^ := pptra^;
                              Dispose(pptra);
                              restoreptr;
                            end
                          else
                            if help1^.nextinst^.instruct = calc_const
                              then
                                begin
                                  help1^.nextinst^.operand := -help1^.nextinst^.operand;
                                  pptr^ := pptra^;
                                  Dispose(pptra);
                                  restoreptr;
                                end;
                      end;
                end
              else  // jetzt werden Operationen mit Konstanten vereinfacht
                if pptr^.instruct in [calc_add..calc_pow]
                  then
                    begin
                      simple(pptra);
                      simple(pptrb);
                      arg1 := pptra^.instruct = calc_const;
                      arg2 := pptrb^.instruct = calc_const;
                      if arg1 and arg2
                        then
                          begin
                            dummy := compute(pptr, pptrb, pptra);
                            pptr^.instruct := calc_const;
                            pptr^.operand := dummy;
                            pptr^.nextinst := pptrb^.nextinst;
                            Dispose(pptra);
                            Dispose(pptrb);
                            restoreptr;
                          end
                        else
                          if arg2
                            then
                              begin
                                if pptrb^.operand = 0.0
                                  then
                                    begin
                                      if pptr^.instruct in [calc_mul.. calc_div, calc_pow]
                                        then
                                          setconst(0.0)
                                        else
                                          if pptr^.instruct = calc_add
                                            then
                                              pusha
                                            else
                                              if pptr^.instruct = calc_sub
                                                then
                                                  begin
                                                    pptr^.instruct := calc_neg;
                                                    help1 := endof(pptra);
                                                    help1^.nextinst := pptrb^.nextinst;
                                                    Dispose(pptrb);
                                                    restoreptr;
                                                  end;
                                    end
                                  else
                                    if (pptrb^.operand = 1.0) and (pptr^.instruct in
                                  [calc_mul, calc_pow])
                                      then
                                        begin
                                          if pptr^.instruct = calc_mul
                                            then pusha
                                            else setconst(1.0);
                                        end;
                              end
                            else
                              if arg1
                                then
                                  begin
                                    if pptra^.operand = 0.0
                                      then
                                        begin
                                          if pptr^.instruct in [calc_mul, calc_pow]
                                            then
                                              begin
                                                if pptr^.instruct = calc_mul
                                                  then dummy := 0.0
                                                  else dummy := 1.0;
                                                pptr^.instruct := calc_const;
                                                pptr^.operand := dummy;
                                                help1 := pptrb;
                                                op := 1;
                                                repeat
                                                  help2 := help1;
                                                  if help1^.instruct in [calc_add.. calc_pow]
                                                    then op := Succ(op)
                                                    else
                                                      if help1^.instruct in [calc_const, calc_var]
                                                        then op := Pred(op);
                                                  help1 := help1^.nextinst;
                                                  Dispose(help2);
                                                until op = 0;
                                                pptr^.nextinst := help1;
                                                Dispose(pptra);
                                                restoreptr;
                                              end
                                            else
                                              if pptr^.instruct in [calc_add, calc_sub]
                                                then skipa;
                                        end
                                      else
                                        if (pptra^.operand = 1.0) and (pptr^.instruct in
                                      [calc_mul..calc_div, calc_pow])
                                          then skipa;
                                  end;
                                    if (pptr^.instruct = calc_mul) and (pptra^.instruct in [calc_div, calc_dvd])
                                      then
                                        begin
                                          help1 := endof(pptra^.nextinst);
                                          if (help1^.nextinst^.instruct = calc_const)
                                            then
                                              if help1^.nextinst^.operand = 1.0
                                                then
                                                  begin
                                                    pptr^ := pptra^;
                                                    Dispose(pptra);
                                                    Dispose(help1^.nextinst);
                                                    help1^.nextinst := pptrb;
                                                    restoreptr;
                                                  end
                                                else
                                                  if help1^.nextinst^.operand = -1.0 then
                                                    begin
                                                      pptr^.instruct := calc_neg;
                                                      Dispose(help1^.nextinst);
                                                      help1^.nextinst := pptrb;
                                                      restoreptr;
                                                    end;
                                        end;
                                    if pptr^.instruct in [calc_mul..calc_div]
                                      then
                                        begin // Negationen vereinfachen
                                          if pptra^.instruct = calc_neg
                                            then
                                              if pptrb^.instruct = calc_neg
                                                then
                                                  begin
                                                    pptr^.nextinst := pptra^.nextinst;
                                                    Dispose(pptra);
                                                    pptra := pptr^.nextinst;
                                                    help1 := endof(pptra);
                                                    help1^.nextinst := pptrb^.nextinst;
                                                    Dispose(pptrb);
                                                    restoreptr;
                                                  end
                                                else
                                                  begin
                                                    if ((pptrb^.instruct = calc_const) and (pptrb^.operand < 0.0))
                                                      then
                                                        begin
                                                          pptr^.nextinst := pptra^.nextinst;
                                                          Dispose(pptra);
                                                          pptrb^.operand := Abs(pptrb^.operand);
                                                          restoreptr;
                                                        end;
                                                  end
                                            else
                                              if ((pptra^.instruct = calc_const) and (pptra^.operand < 0.0))
                                                then
                                                  if pptrb^.instruct = calc_neg
                                                    then
                                                      begin
                                                        pptra^.nextinst := pptrb^.nextinst;
                                                        Dispose(pptrb);
                                                        pptra^.operand := Abs(pptra^.operand);
                                                        restoreptr;
                                                      end;
                                          if (pptra^.instruct = calc_const) and (pptra^.operand = -1.0)
                                            then
                                              begin
                                                pptr^.instruct := calc_neg;
                                                pptr^.nextinst := pptrb;
                                                Dispose(pptra);
                                                restoreptr;
                                              end;
                                          if ((pptrb^.instruct = calc_const) and (pptrb^.operand = -1.0) and
                                            (pptr^.instruct = calc_mul))
                                            then
                                              begin
                                                help1 := endof(pptra);
                                                help1^.nextinst := pptrb^.nextinst;
                                                pptr^.instruct := calc_neg;
                                                Dispose(pptrb);
                                                restoreptr;
                                              end;
                                        end;
                                    if (pptr^.instruct = calc_add) and (pptra^.instruct = calc_neg)
                                      then
                                        begin
                                          pptr^.instruct := calc_sub;
                                          pptr^.nextinst := pptra^.nextinst;
                                          Dispose(pptra);
                                          restoreptr;
                                        end;
                                    if (pptr^.instruct = calc_sub) and (pptra^.instruct = calc_neg)
                                      then
                                        begin
                                          pptr^.instruct := calc_add;
                                          pptr^.nextinst := pptra^.nextinst;
                                          Dispose(pptra);
                                          restoreptr;
                                        end;
                                              // jetzt wird's schwierig : das Kommutativgesetz
                                    if (((pptr^.instruct = calc_mul) and (pptra^.instruct in
                                      [calc_mul..calc_div])) or ((pptr^.instruct = calc_add) and
                                      (pptra^.instruct in [calc_add, calc_sub]))) and
                                      (pptrb^.instruct = calc_const)
                                      then
                                        begin
                                          help1 := endof(pptra^.nextinst);
                                          help2 := endof(help1^.nextinst);
                                          if pptra^.instruct in [calc_mul, calc_add]
                                            then
                                              begin
                                                if help1^.nextinst^.instruct = calc_const
                                                  then
                                                    begin
                                                      help2^.nextinst := pptra^.nextinst;
                                                      pptra^.nextinst := pptrb;
                                                      help2 := pptrb^.nextinst;
                                                      pptrb^.nextinst := help1^.nextinst;
                                                      help1^.nextinst := help2;
                                                    end
                                                  else
                                                    if pptra^.nextinst^.instruct = calc_const
                                                      then
                                                        begin
                                                          help2^.nextinst := pptrb^.nextinst;
                                                          pptrb^.nextinst := help1^.nextinst;
                                                          help1^.nextinst := pptrb;
                                                        end;
                                              end
                                            else
                                              begin
                                                if help1^.nextinst^.instruct = calc_const
                                                  then
                                                    begin
                                                      helpinstruct := pptr^.instruct;
                                                      pptr^.instruct := pptra^.instruct;
                                                      pptra^.instruct := helpinstruct;
                                                      pptr^.nextinst := pptra^.nextinst;
                                                      pptra^.nextinst := help1^.nextinst;
                                                      help1^.nextinst := pptra;
                                                    end;
                                              end;
                                          restoreptr;
                                          simple(pptra);
                                          simple(pptrb);
                                        end
                                      else
                                        if (((pptr^.instruct = calc_mul) and (pptrb^.instruct in
                                           [calc_mul..calc_div])) or ((pptr^.instruct = calc_add) and
                                           (pptrb^.instruct in [calc_add, calc_sub]))) and
                                           (pptra^.instruct = calc_const)
                                          then
                                            begin
                                              help1 := endof(pptrb^.nextinst);
                                              help2 := endof(help1^.nextinst);
                                              if pptrb^.instruct in [calc_add, calc_mul]
                                                then
                                                  begin
                                                    if pptrb^.nextinst^.instruct = calc_const
                                                      then
                                                        begin
                                                          pptr^.nextinst := help1^.nextinst;
                                                          help1^.nextinst := pptra;
                                                          pptra^.nextinst := help2^.nextinst;
                                                          help2^.nextinst := pptrb;
                                                        end
                                                      else
                                                        if help1^.nextinst^.instruct = calc_const
                                                          then
                                                            begin
                                                              pptr^.nextinst := pptrb^.nextinst;
                                                              pptra^.nextinst := help1^.nextinst;
                                                              help1^.nextinst := pptrb;
                                                              pptrb^.nextinst := pptra;
                                                            end;
                                                end
                                              else
                                                begin
                                                  if help1^.nextinst^.instruct = calc_const
                                                    then
                                                      begin
                                                        helpinstruct := pptr^.instruct;
                                                        pptr^.instruct := pptrb^.instruct;
                                                        pptrb^.instruct := helpinstruct;
                                                        pptr^.nextinst := pptrb^.nextinst;
                                                        pptrb^.nextinst := pptra;
                                                        pptra^.nextinst := help1^.nextinst;
                                                        help1^.nextinst := pptrb;
                                                      end;
                                                end;
                                              restoreptr;
                                              simple(pptra);
                                              simple(pptrb);
                                            end;
                                              if pptr^.instruct = calc_mul then
                                              begin
                                                if pptra^.instruct = calc_pow then
                                                begin
                                                  help2 := pptra^.nextinst;
                                                  help1 := endof(help2);
                                                  help1 := help1^.nextinst;
                                                  if (help2^.instruct = calc_const) and equal(help1, pptrb) then
                                                  begin
                                                    help2^.operand := help2^.operand + 1.0;
                                                    pptr^ := pptra^;
                                                    Dispose(pptra);
                                                    help2^.nextinst := pptrb;
                                                    erase_entry;
                                                    restoreptr;
                                                  end;
                                                end;
                                              end;
                                    if pptr^.instruct in [calc_add, calc_sub, calc_dvd, calc_div, calc_mul]
                                      then
                                        if equal(pptra, pptrb)
                                          then // sind die Operanden gleich ?
                                            begin
                                              case pptr^.instruct of
                                                calc_add: begin
                                                            pptr^.instruct := calc_mul;
                                                            help2 := endof(pptra);
                                                            help1 := pptra^.nextinst;
                                                            pptra^.instruct := calc_const;
                                                            pptra^.operand := 2.0;
                                                            pptra^.nextinst := help2^.nextinst;
                                                            erase_entry;
                                                            restoreptr;
                                                          end;
                                                calc_sub: setconst(0.0);
                                                calc_div,
                                                calc_dvd: setconst(1.0);
                                                calc_mul: begin
                                                            pptr^.instruct := calc_pow;
                                                            help2 := endof(pptra);
                                                            help1 := pptra^.nextinst;
                                                            pptra^.nextinst := help2^.nextinst;
                                                            pptra^.instruct := calc_const;
                                                            pptra^.operand := 2.0;
                                                            erase_entry;
                                                            restoreptr;
                                                          end;
                                              end; { case }
                                            end;  { if equal }
                    end; { pptr^.instruct in [calc_add..calc_pow] }
          end; { pptr <> nil }
    end; // Simple

  begin // CalcSimplify
    if pptr <> nil
     then
       begin
         SimpleError := False;
         CalcResult := True;
         Invert(pptr);
         pptr1 := pptr^.nextinst;
         Simple(pptr1);
         if SimpleError
           then
             begin
               KillExpression(pptr);
               CalcResult := False;
             end
           else
             invert(pptr);
       end
     else
      calcresult := False;
  end; // CalcSimlify
\end{lstlisting}

\subsection{Derivative of a function}

\begin{lstlisting}
  function CalcDerivation(pptr: calc_prog; vartab: calc_vartab;
                          nach: calc_idstr): calc_prog;

  const
    pi_durch_180 = 1.745329252e-2;

  var
    pptrstart, pptr1 : calc_prog;
    ok               : boolean;


    procedure UpperCase(var varid: calc_idstr);

    var
      i: integer;

    begin
      for i := 1 to Length(varid) do
        varid[i] := Upcase(varid[i]);
    end;


    procedure NewConst(x: calc_operand);

    var
      pptr : calc_prog;

    begin
      try
        begin
          New(pptr);
          pptr^.instruct := calc_const;
          pptr^.operand := x;
          pptr^.nextinst := pptrstart^.nextinst;
          pptrstart^.nextinst := pptr;
        end
      except
          calcresult := False;
      end;
    end;


    procedure NewOp(id: calc_symbols);

    var
      pptr: calc_prog;

    begin
      try
        begin
          New(pptr);
          pptr^.instruct := id;
          pptr^.nextinst := pptrstart^.nextinst;
          pptrstart^.nextinst := pptr;
        end
      except
        calcresult := False;
      end;
    end;


    procedure push(pptr: calc_prog);

    var
      pptr1: calc_prog;
      op: integer;

    begin
      op := 1;
      repeat
        try
          begin
            if pptr^.instruct in [calc_add..calc_pow]
              then op := op + 1
              else
                if not (pptr^.instruct in [calc_neg, calc_sqr..calc_fak])
                  then
                    op := op - 1;
            New(pptr1);
            pptr1^ := pptr^;
            pptr1^.nextinst := pptrstart^.nextinst;
            pptrstart^.nextinst := pptr1;
            pptr := pptr^.nextinst;
          end
        except
          calcresult := False
      until (op = 0) or not calcresult;
    end; // Push


    procedure derive(pptr : calc_prog);

    var
      pptra, pptrb : calc_prog;

    begin
      if calcresult
        then
          begin
            pptra := pptr^.nextinst;
            if (pptra <> nil)
              then
                begin
                  pptrb := endof(pptra);
                  pptrb := pptrb^.nextinst;
                end;
            case pptr^.instruct of
              calc_neg:  begin
                           newop(calc_neg);
                           derive(pptra);
                         end;
              calc_const,
              calc_div..calc_kgv,
              calc_fak : begin
                           newconst(0.0);
                         end;
              calc_var : begin
                           if nach = vartab^[pptr^.varindex].varid
                             then newconst(1.0)
                             else newconst(0.0);
                         end;
              calc_add : begin
                           if calc_const in [pptra^.instruct, pptrb^.instruct]
                             then
                               if pptra^.instruct = calc_const
                                 then derive(pptrb)
                                 else derive(pptra)
                             else
                               begin
                                 newop(calc_add);
                                 derive(pptra);
                                 derive(pptrb);
                               end;
                         end;
              calc_sub : begin
                           if calc_const in [pptra^.instruct, pptrb^.instruct]
                             then
                               if pptra^.instruct = calc_const
                                 then derive(pptrb)
                                 else
                                   begin
                                     newop(calc_neg);
                                     derive(pptra);
                                   end
                             else
                               begin
                                 newop(calc_sub);
                                 derive(pptra);
                                 derive(pptrb);
                               end;
                         end;
              calc_mul : begin
                           if calc_const in [pptra^.instruct, pptrb^.instruct]
                             then
                               if pptra^.instruct = calc_const
                                 then
                                   begin
                                     newop(calc_mul);
                                     push(pptra);
                                     derive(pptrb);
                                   end
                                 else
                                   begin
                                     newop(calc_mul);
                                     push(pptrb);
                                     derive(pptra);
                                   end
                             else
                               begin
                                 newop(calc_add);
                                 newop(calc_mul);
                                 derive(pptra);
                                 push(pptrb);
                                 newop(calc_mul);
                                 push(pptra);
                                 derive(pptrb);
                               end;
                         end;
              calc_dvd: begin
                          if pptra^.instruct = calc_const
                            then
                              begin
                                newop(calc_dvd);
                                push(pptra);
                                derive(pptrb);
                              end
                            else
                              begin
                                newop(calc_dvd);
                                newop(calc_sqr);
                                push(pptra);
                                newop(calc_sub);
                                newop(calc_mul);
                                derive(pptra);
                                push(pptrb);
                                newop(calc_mul);
                                push(pptra);
                                derive(pptrb);
                              end;
                        end;
              calc_pow : begin
                           if (pptrb^.instruct = calc_const) and (pptrb^.operand < 0.0)
                             then calcresult := False
                             else
                               begin
                                 ok := False;
                                 case pptra^.instruct of
                                   calc_const : ok := True;
                                   calc_var   : ok := nach <> vartab^[pptra^.varindex].varid
                                 end;
                                 if ok then
                                   begin
                                     newop(calc_mul);
                                     newop(calc_mul);
                                     newop(calc_pow);
                                     newop(calc_sub);
                                     newconst(1.0);
                                     push(pptra);
                                     push(pptrb);
                                     push(pptra);
                                     derive(pptrb);
                                   end
                                 else
                                   begin
                                     newop(calc_mul);
                                     newop(calc_pow);
                                     push(pptra);
                                     push(pptrb);
                                     newop(calc_add);
                                     newop(calc_dvd);
                                     push(pptrb);
                                     newop(calc_mul);
                                     push(pptra);
                                     derive(pptrb);
                                     newop(calc_mul);
                                     derive(pptra);
                                     newop(calc_ln);
                                     push(pptrb);
                                   end;
                              end;
                         end;
              calc_abs : begin
                           newop(calc_mul);
                           newop(calc_sign);       // calc_sig ???
                           push(pptra);
                           derive(pptra);
                         end;
              calc_int,
              calc_sign: newconst(0.0);
              calc_sqr : begin
                           newop(calc_mul);
                           newop(calc_mul);
                           push(pptra);
                           derive(pptra);
                           newconst(2.0);
                         end;
              calc_sqrt: begin
                           newop(calc_dvd);
                           newop(calc_mul);
                           newconst(2.0);
                           newop(calc_sqrt);
                           push(pptra);
                           derive(pptra);
                         end;
              calc_exp : begin
                           newop(calc_mul);
                           newop(calc_exp);
                           push(pptra);
                           derive(pptra);
                         end;
              calc_ln : begin
                          newop(calc_dvd);
                          push(pptra);
                          derive(pptra);
                        end;
              calc_lg : begin
                          newop(calc_dvd);
                          newop(calc_mul);
                          newop(calc_ln);
                          newconst(10.0);
                          push(pptra);
                          derive(pptra);
                        end;
              calc_ld : begin
                          newop(calc_dvd);
                          newop(calc_mul);
                          newop(calc_ln);
                          newconst(2.0);
                          push(pptra);
                          derive(pptra);
                        end;
              calc_sin : begin
                          newop(calc_mul);
                          newop(calc_cos);
                          push(pptra);
                          derive(pptra);
                        end;
              calc_cos : begin
                          newop(calc_mul);
                          newop(calc_neg);
                          newop(calc_sin);
                          push(pptra);
                          derive(pptra);
                        end;
              calc_tan : begin
                          newop(calc_dvd);
                          newop(calc_sqr);
                          newop(calc_cos);
                          push(pptra);
                          derive(pptra);
                        end;
              calc_cot: begin
                          newop(calc_neg);
                          newop(calc_dvd);
                          newop(calc_sqr);
                          newop(calc_sin);
                          push(pptra);
                          derive(pptra);
                        end;
              calc_arcsin : begin
                          newop(calc_dvd);
                          newop(calc_sqrt);
                          newop(calc_sub);
                          newop(calc_sqr);
                          push(pptra);
                          newconst(1.0);
                          derive(pptra);
                        end;
              calc_arccos : begin
                          newop(calc_neg);
                          newop(calc_dvd);
                          newop(calc_sqrt);
                          newop(calc_sub);
                          newop(calc_sqr);
                          push(pptra);
                          newconst(1.0);
                          derive(pptra);
                        end;
              calc_arctan: begin
                          newop(calc_dvd);
                          newop(calc_add);
                          newop(calc_sqr);
                          push(pptra);
                          newconst(1.0);
                          derive(pptra);
                        end;
              calc_arccot : begin
                          newop(calc_neg);
                          newop(calc_dvd);
                          newop(calc_add);
                          newop(calc_sqr);
                          push(pptra);
                          newconst(1.0);
                          derive(pptra);
                        end;
              calc_sinh : begin
                          newop(calc_mul);
                          newop(calc_cosh);
                          push(pptra);
                          derive(pptra);
                        end;
              calc_cosh : begin
                          newop(calc_mul);
                          newop(calc_sinh);
                          push(pptra);
                          derive(pptra);
                        end;
              calc_tanh : begin
                          newop(calc_dvd);
                          newop(calc_sqr);
                          newop(calc_cosh);
                          push(pptra);
                          derive(pptra);
                        end;
              calc_coth : begin
                          newop(calc_neg);
                          newop(calc_dvd);
                          newop(calc_sqr);
                          newop(calc_sinh);
                          push(pptra);
                          derive(pptra);
                        end;
              calc_arcsinh : begin
                          newop(calc_dvd);
                          newop(calc_sqrt);
                          newop(calc_add);
                          newconst(1.0);
                          newop(calc_sqr);
                          push(pptra);
                          derive(pptra);
                        end;
              calc_arccosh : begin
                          newop(calc_dvd);
                          newop(calc_sqrt);
                          newop(calc_sub);
                          newconst(1.0);
                          newop(calc_sqr);
                          push(pptra);
                          derive(pptra);
                        end;
              calc_arctanh,
              calc_arccoth : begin
                          newop(calc_dvd);
                          newop(calc_sub);
                          newop(calc_sqr);
                          push(pptra);
                          newconst(1.0);
                          derive(pptra);
                        end;
              calc_deg : begin
                          newop(calc_dvd);
                          newconst(pi_durch_180);
                          derive(pptra);
                        end;
              calc_rad: begin
                          newop(calc_mul);
                          newconst(pi_durch_180);
                          derive(pptra);
                        end;
              else      calcresult := False
            end; // case
        end; // if CalcResult
    end; // Derive

  begin // CalcDerivation
    if pptr <> nil
      then
        begin
          uppercase(nach);
          invert(pptr);
          pptr1 := pptr;
          New(pptrstart);
          pptrstart^.nextinst := nil;
          pptr := pptr^.nextinst;
          calcresult := True;
          derive(pptr);
          if calcresult then
          begin
            calcsimplify(pptrstart);
            Result := pptrstart;
          end
          else
          begin
            killexpression(pptrstart);
            Result := nil;
            CalcError(4, 'Derivate of function not known');
          end;
          invert(pptr1);
        end
      else
        begin
          Result := nil;
          CalcResult := False;
        end;
  end; // CalcDerivation
\end{lstlisting}

\subsection{Compile the function}

\begin{lstlisting}[caption=]
  procedure CompileExpression(Expr: Calc_String; var VarTable: Calc_VarTab;
    var ExprPtr: Calc_Prog);

  var
    VarTabFlag, ParsError, EndOfExpr : boolean;
    ch                               : string[1];    // akt. Zeichen aus String
    LastPos, StrPos                  : integer;      // zaehlt String-Position mit
    TempIdent, Ident                 : Calc_String;  // enth. aktuellen Bezeichner
    Symbol,                                          // akt. Symbol des Bezeichners
    LastSymbol                       : Calc_Symbols; // vorheriges Symbol
    Number                           : Calc_Operand; // akt. Zahl/Index aus String
    ProgPtr                          : Calc_Prog;


    procedure Error(ErrPos: integer; ErrMsg: Calc_String);

    begin
      if not ParsError
        then
          begin
            WriteLn;
            Write('*** ', ' Error compiling this expression: ');
            ClrEol;
            WriteLn;
            Write(' ', Expr);
            ClrEol;
            Writeln;
            Write(' ': ErrPos, '^');
            ClrEol;
            Writeln;
            Write(ErrMsg, '!');
            Clreol;
            Writeln;
            ClrEol;
            WriteLn;
            ClrEol;
          end;
      ParsError := True;
      Symbol := Calc_Err;
    end;


    procedure Add_To_Queue(op: Calc_Symbols; x: Calc_Operand);
    { add next operand to que }

    var
      UPN_Entry: Calc_Prog;

    begin
      try  // is enough memory available?
        begin
          New(UPN_Entry);
          with UPN_Entry^ do
            begin
              NextInst := nil;
              Instruct := op;
              case op of
                Calc_Var   : VarIndex := Trunc(x);
                Calc_Const : Operand  := x
              end;
            end;
          ProgPtr^.NextInst := UPN_Entry;
          ProgPtr := ProgPtr^.NextInst;
        end
      except
        Error(1, 'not enough free memory');
      end;
    end;


    procedure GetSymbol;
    { get next symbol from string }


      procedure GetChar;
      { get next character from string }

      begin
        ch := ' ';
        StrPos := Succ(StrPos);
        EndOfExpr := (StrPos > Length(Expr));
        if not EndOfExpr
          then ch := UpCase(Expr[StrPos]);
      end;


      procedure GetNumber;
      { Get the next number from string. The Turbo-Pascal val-procedure wants
        to see only valid characters of a floating point number, NumberEnd
        points to the first invalid character. Hence:                          }

      var
        NumberStr: Calc_String;
        NumberEnd, posi: integer;

      begin
        NumberStr := Copy(Expr, StrPos, 255); // everything from first figure
        NumberStr := NumberStr + '     ';
        posi := 1;
        while (not (numberstr[posi] in ['e', 'E'])) and (posi < length(numberstr)) do
          posi := succ(posi);
        if numberstr[posi] in ['e', 'E']
          then
            begin
              if numberstr[posi + 1] in ['+', '-']
                then posi := succ(posi);
              if not (numberstr[posi + 1] in ['0'..'9'])
                then error(StrPos + posi, 'incomplete expression')
                else
                  if ((numberstr[posi + 1] = '3') and (numberstr[posi + 2] in
                      ['7'..'9'])) or ((numberstr[posi + 1] > '3') and
                      (numberstr[posi + 2] in ['0'..'9']))
                    then error(StrPos + posi, 'number out of range');
            end;
        Val(NumberStr, Number, NumberEnd);
        if NumberEnd > 0
          then                        // invalid character
            begin                     // number not at the end of expression
             StrPos := StrPos + NumberEnd - 2;
             NumberStr := Copy(NumberStr, 1, Pred(NumberEnd));
             Val(NumberStr, Number, NumberEnd);
           end
         else                         // worked, at the end of expression
          StrPos := Length(Expr);
      end;


      procedure searchsymtab;
      { look up symbol in symbol table }

      var
        symok: boolean;

      begin
        symok := False;
        symbol := calc_err;
        while (symbol < calc_end) and not symok do
          begin
            symbol := Succ(symbol);
            symok := (ident = calc_ids[symbol]);
          end;
        if not symok
          then symbol := calc_err;
      end;

    begin  // GetSymbol
      LastPos := StrPos;
      LastSymbol := Symbol;
      Ident := '';
      while (ch = ' ') and not EndOfExpr do
        GetChar;
      case ch[1] of
        'A'..'Z': repeat      // Name of operator, function or variable
                    Ident := Concat(Ident, ch);
                    GetChar
                  until not (ch[1] in ['A'..'Z', '0'..'9']);
        '0'..'9', '.':
                  begin
                    GetNumber;
                    Ident := 'CONST';
                    GetChar;
                  end
        else      begin          // +, -, etc.
                    Ident := ch;
                    GetChar;
                  end
      end;
      SearchSymtab;
      if Symbol = Calc_Err
        then  // operator not identified
          if Ident[1] in ['A'..'Z']
            then Symbol := Calc_Var  // use as variable
            else
              if Ident <> ' '
                then Error(LastPos, 'unknown symbol');
      if Symbol = Calc_Var
        then
          if LastSymbol <> Calc_Var
            then
              begin
               Number := SearchVarTab(VarTable, Ident);
               if CalcDecMod and (Number = 0.0)
                 then  // refuse unidentified string
                   Error(LastPos, 'unknown symbol')
                 else  // or accept it as new var
                   if Number = 0.0
                     then
                       begin
                         Number := AddToVarTab(VarTable, Ident);
                         if Number < 0.0
                           then Error(LastPos, 'too many variables');
                       end;
              end
        else
          if not EndOfExpr
            then Error(LastPos, 'operator expected');
    end; // GetSymbol


    procedure Expression;
    { expression contains several parts connected by operators }

    var
      ExprOp: Calc_Symbols;

      procedure Term;
      { expression contains several factors }

      var
        TermOp: Calc_Symbols;

        procedure Factor(fparen: boolean);
        { Factors can be variables, constants, functions or an expression in
          brackets, and all of them can be raised to a power.
          The parameter 'fparen' indicates whether brackets are for an expression
          or a function. In the latter case, the function needs to be evaluated
          before raising to power                                                }

         var
          FacOp: Calc_Symbols;

        begin
          if Symbol <> Calc_Err
            then
              begin
                case Symbol of
                  Calc_Var   : begin
                                 Add_To_Queue(Calc_Var, Number);
                                 GetSymbol;
                               end;
                  Calc_Const : begin
                                 Add_To_Queue(Calc_Const, Number);
                                 GetSymbol;
                               end;
                  Calc_Pi    : begin
                                 Add_To_Queue(Calc_Const, Const_pi);
                                 GetSymbol;
                               end;
                  Calc_E     : begin
                                 Add_To_Queue(Calc_Const, Const_e);
                                 GetSymbol;
                               end;
                  Calc_lp    : begin                 // expression in brackets
                                 GetSymbol;
                                 Expression;
                                 if (Symbol <> Calc_rp)
                                   then Error(StrPos - Ord(ch[1] > ' '),
                                              Concat(Calc_Ids[Calc_rp], ' expected'));
                                 GetSymbol;
                               end;
                  Calc_Pow   : ;      // power, dealt with later
                  Calc_Sqr..
                  Calc_Fak   : begin  // funktion
                                 FacOp := Symbol;
                                 GetSymbol;
                                 if (Symbol = Calc_lp)
                                   then Factor(True)
                                   else Error(LastPos, Concat(Calc_Ids[Calc_lp], ' expected'));
                                 Add_To_Queue(FacOp, 0);
                               end
                  else         Error(LastPos, 'here unexpected')
                end; // CASE
                if not fparen
                  then // brackets notfor function, therefore..
                    if Symbol = Calc_Pow
                      then  // power
                        if LastSymbol in [Calc_Const, Calc_Var, Calc_rp, Calc_PI, Calc_e]
                          then
                            begin  // evaluate
                              GetSymbol;
                              Factor(False);
                              Add_To_Queue(Calc_Pow, 0);
                            end
                          else      // power here not possible
                            Error(Pred(StrPos), 'here unexpected');
              end // IF Symbol <> Calc_Err
            else
              Error(StrPos - Ord(EndOfExpr), ' incomplete expression');
        end; // Faktor

      begin  // Term
        Factor(False);
        if Symbol in [Calc_Mul..Calc_kgV]
          then  // term contains several factors
            begin
              TermOp := Symbol;
              GetSymbol;
              Term;
              Add_To_Queue(TermOp, 0);
            end;
      end; // Term

    begin  // Expression
      if Symbol in [Calc_Add..Calc_Sub]
        then // expression starts with + or -
          begin
            ExprOp := Symbol;
            GetSymbol;
            Term;
            if ExprOp = Calc_Sub
              then Add_To_Queue(Calc_Neg, 0);      // negate everything
          end
        else
          Term;
      while (Symbol in [Calc_Add..Calc_Sub]) do
        begin                              // additional terms follow
          ExprOp := Symbol;
          GetSymbol;
          Term;
          Add_To_Queue(ExprOp, 0);
        end;
    end; // Expression

  begin  // CompileExpression
    Symbol := Calc_Err;
    ParsError := False;
    EndOfExpr := False;
    StrPos := 0;
    ch := ' ';
    VarTabFlag := (VarTable = nil);
    if VarTabFlag
      then VarTable := NewVarTab;
    New(ExprPtr);
    ExprPtr^.NextInst := nil;
    ProgPtr := ExprPtr;
    GetSymbol;
    Expression;
    if Symbol <> Calc_EOE
      then Error(LastPos, '";" ' + ' expected');
    CalcResult := not ParsError;
    if ParsError
      then
        begin
          KillExpression(ExprPtr);
          if VarTabFlag
            then KillVarTab(VarTable);
        end;
  end; // CompileExpression
\end{lstlisting}

\subsection{Calculate value of an expression}

\begin{lstlisting}
  function CalcExpression(ExprPtr: Calc_Prog; VarTable: Calc_VarTab): Calc_Operand;
  { evaluate an RPN-expression  }

  const
    StackSize = 50;

  var
    x        : Calc_Operand;
    StackPtr : integer;
    Stack    : array[1..StackSize] of Calc_Operand;


    procedure Push;                        // pushes number onto the stack

    begin
      Stack[StackPtr] := x;
      StackPtr := Succ(StackPtr);
    end;


    function Pop: Calc_Operand;            // gets a number from stack

    begin
      StackPtr := Pred(StackPtr);
      Result := Stack[StackPtr];
    end;

  begin  // CalcExpression
    CalcResult := True;
    if (ExprPtr <> nil) and (VarTable <> nil)
      then
        begin
          ExprPtr := ExprPtr^.NextInst;
          StackPtr := 1;
          x := 0.0;
          while ExprPtr <> nil do
            begin
              with ExprPtr^ do
                case Instruct of
                  Calc_Const : begin
                                 Push;
                                 x := Operand;
                               end;
                  Calc_Var   : begin
                                 Push;
                                 x := VarTable^[VarIndex].Value;
                               end;
                  else         begin
                                case Instruct of
                                  Calc_Neg   : x := -x;
                                  Calc_Add   : x := Pop + x;
                                  Calc_Sub   : x := Pop - x;
                                  Calc_Mul   : x := Pop * x;
                                  Calc_Dvd   : if x <> 0.0
                                                 then x := Pop / x
                                                 else CalcError(2, '');
                                  Calc_Div   : if Trunc(x) <> 0
                                               then x := Trunc(Pop) div Trunc(x)
                                               else CalcError(2, '');
                                  Calc_Mod   : if Trunc(x) <> 0
                                                 then x := Trunc(Pop) mod Trunc(x)
                                                 else CalcError(2, '');
                                  Calc_ggT   : x := GCD(Trunc(Pop), Trunc(x));
                                  Calc_kgV   : x := SCM(Trunc(Pop), Trunc(x));
                                  Calc_Pow   : x := pot(Pop, x);
                                  Calc_Sqr   : x := Sqr(x);
                                  Calc_Sqrt  : if x >= 0.0
                                                 then x := Sqrt(x)
                                                 else CalcError(3, 'Sqrt(x): x <= 0');
                                  Calc_Exp   : x := Exp(x);
                                  Calc_Ln    : if x > 0.0
                                                 then x := Ln(x)
                                                 else CalcError(3, 'ln(x): x <= 0');
                                  Calc_Lg    : x := log(x, 10);
                                  Calc_Ld    : x := log(x, 2);
                                  Calc_Sin   : x := Sin(x);
                                  Calc_Cos   : x := Cos(x);
                                  Calc_Tan   : x := tan(x);
                                  Calc_Cot   : x := cot(x);
                                  Calc_ArcSin: x := arcsin(x);
                                  Calc_ArcCos: x := arccos(x);
                                  Calc_ArcTan: x := ArcTan(x);
                                  Calc_ArcCot: x := arccot(x);
                                  Calc_Sinh  : x := sinh(x);
                                  Calc_Cosh  : x := cosh(x);
                                  Calc_Tanh  : x := tanh(x);
                                  Calc_Coth  : x := coth(x);
                                  Calc_ArcSinh: x := arsinh(x);
                                  Calc_ArcCosh: x := arcosh(x);
                                  Calc_ArcTanh: x := artanh(x);
                                  Calc_ArcCoth: x := arcoth(x);
                                  Calc_Abs   : x := abs(x);
                                  Calc_Deg   : x := grad(x);
                                  Calc_Rad   : x := rad(x);
                                  Calc_Rez   : if x <> 0
                                                 then x := 1 / x
                                                 else CalcError(3, '1 / 0');
                                  Calc_Fak   : x := fak(Round(x));
                                  Calc_Int   : x := Int(x);
                                  Calc_Sign  : x := Signum(x);
                                  else         CalcError(0, 'Function not known')
                                 end; // CASE
                               end; // ELSE
                end; // CASE
              ExprPtr := ExprPtr^.NextInst;
              if StackPtr > StackSize then CalcError(0, 'Stack overflow');
              if not CalcResult then ExprPtr := nil;
            end; // WHILE
          Result := x;
        end // THEN
    else
      CalcError(0, 'Function not known');
  end;
\end{lstlisting}

\subsection{Screen output of calc programs}

The recursive procedure \texttt{CalcAOS} converts RPN-programs into ``normal'' formulas for screen output.

\begin{lstlisting}
  procedure CalcAOS(pptr: Calc_Prog; VarTable: Calc_VarTab);

  var
    Value   : string[50];
    len     : byte ABSOLUTE Value;
    key     : char;
    dummy   : calc_operand;
    pptr1   : calc_prog;


    procedure writeaos(pptr: calc_prog);

    var
      pptra, pptrb : calc_prog;
      paren        : boolean;

    begin
      if (pptr <> nil)
        then
          begin
            if keypressed
              then
                begin
                  key := ReadKey;
                  if key = ^s then Key := readkey;
                end;
            pptra := pptr^.nextinst;
            if pptra <> nil
              then
                begin
                  pptrb := endof(pptra);
                  pptrb := pptrb^.nextinst;
                end;
            case pptr^.instruct of
              calc_const: begin
                            dummy := pptr^.operand;
                            Str(dummy: 0: 10, Value);
                            while Value[len] = '0' do
                              len := Pred(len);
                            if Value[len] = '.' then len := Pred(len);
                            paren := dummy < 0.0;
                            if paren then Write('(');
                            Write(Value);
                            if paren then Write(')');
                          end;
              calc_var  : Write(vartable^[pptr^.varindex].varid);
              calc_add..
              calc_pow  : begin
                            paren := (pptr^.instruct in [calc_mul..calc_pow]) and
                                     (pptrb^.instruct in [calc_add, calc_sub]);
                            paren := paren or (pptr^.instruct = calc_pow) and
                                     (pptrb^.instruct in [calc_add..calc_pow]);
                            if paren then Write('(');
                            writeaos(pptrb);
                            if paren then Write(')');
                            if pptr^.instruct in [calc_div..Calc_Kgv] then Write(' ');
                            Write(calc_ids[pptr^.instruct]);
                            if pptr^.instruct in [calc_div..Calc_Kgv] then Write(' ');
                            paren := (pptr^.instruct in [calc_mul..calc_pow]) and
                                     (pptra^.instruct in [calc_add, calc_sub]);
                            paren := paren or ((pptr^.instruct in [calc_dvd..calc_pow])
                                     and (pptra^.instruct in [calc_add..calc_pow]));
                            paren := paren or ((pptr^.instruct = calc_sub) and
                                     (pptra^.instruct in [calc_add, calc_sub]));
                            if paren then Write('(');
                            writeaos(pptra);
                            if paren then Write(')');
                          end;
              calc_neg  : begin
                            Write('(-');
                            paren := pptra^.instruct in [calc_add..calc_pow];
                            if paren then Write('(');
                            writeaos(pptra);
                            if paren then Write(')');
                            Write(')');
                          end;
              calc_sqr..
              calc_fak  : begin
                            Write(calc_ids[pptr^.instruct], '(');
                            writeaos(pptra);
                            Write(')');
                          end
              end; // case
          end;     // then
    end;           // WriteAOS

  begin            // CalcAOS
    if pptr <> nil
      then
        begin
          pptr1 := pptr;
          invert(pptr);
          pptr := pptr^.nextinst;
          writeaos(pptr);
          invert(pptr1);
          Writeln;
        end
      else
        Writeln('Function not defined');
  end; // CalcAOS


  end.            // Unit Calc
\end{lstlisting}


\subsection{Test program}

\begin{lstlisting}
  PROGRAM CalcDemo;

  USES Calc, Vector;

  VAR a, b, dx               : REAL;
      dummy                  : REAL;
      Formula                : Calc_String;
      FormProg               : Calc_Prog;
      x                      : Calc_VarTab;
      f, xv, yv, x1, x2      : VectorTyp;
      xmin, xmax, ymin, ymax : REAL;
      n                      : WORD;
      i                      : INTEGER;
      c                      : STRING[1];

  BEGIN
    x := NewVarTab;
    dummy := AddToVarTab(x, 'X');
    CalcDecMod := TRUE;
    REPEAT
      WriteLn;
      WriteLn;
      WriteLn('Please enter function to be evaluated: ');
      WriteLn;
      Write('f(x) = '); ReadLn(Formula); WriteLn;
      CompileExpression(Formula, x, FormProg);
      IF CalcResult
        THEN
          BEGIN
            WriteLn('Expression "', Formula, '" compiled correctly...');
            WriteLn;
            Write('Evaluate f(x) between  a = '); Read(a);
            Write('                   and b = '); Read(b);
            Write('      with step width dx = '); ReadLn(dx);
            WriteLn;
            i := Round((abs(a-b)/dx));
            FOR n := 1 TO i+1 DO
              BEGIN
                SetVectorElement(xv, n, a + dx);
                AssignVar(x, 'X', a);
                SetVectorElement(yv, n, CalcExpression(FormProg, x));
                a := a + dx;
              END;
            KillExpression(FormProg);
        END;
    UNTIL Formula = '';
    KillVarTab(x);
    WriteLn('Demo finished...');
  END.
\end{lstlisting}

\printbibliography[heading=subbibliography]
\end{refsection}
